\documentclass{article}
\usepackage{graphicx} % Required for inserting images
\usepackage{algpseudocode}
\usepackage{algorithm}
\usepackage{amsmath}


\title{Binary Search}
\author{Dimael Rivas}
\date{July 2024}

\begin{document}
	
	\maketitle
	
	\section{Algorithm}
	Binary search is an algorithm that finds an element in a sorted array in $\mathcal{O}(\log{}n)$, where n is the size of the array. This algorithm works by asking each time the middle of the array, so we get rid of a half of the array on each query.
	
	In a general way, this algorithm works if the following property is satisfied:
	$$T(i) \leq T(i+1)$$
	That means the data is divided into two parts, such that one part satisfies function T(i) and the other do not.
	
	\begin{algorithm}
		\caption{Binary Search}
		\begin{algorithmic}[0]
			\Require $l, r, T$
			\Ensure $ans$
			\While {$l \leq r $}
			\State $mid \gets l + \frac{r-l}{2}$
			\If{T(mid) is true}
			\State $r \gets mid$
			\Else
			\State $l \gets mid$
			\EndIf
			\EndWhile
			\State $ans \gets T(l)$
		\end{algorithmic}
	\end{algorithm}
	This pseudo-code may vary, because it all depends on the T function.
	\section{Classic Problems}
	
\end{document}
