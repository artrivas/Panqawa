\documentclass[10pt,a4paper]{article}
\usepackage[utf8]{inputenc}
\usepackage[T1]{fontenc}
\usepackage{amsmath}
\usepackage{amssymb}
\usepackage{graphicx}
\usepackage[english]{babel}
\usepackage{algpseudocode}
\usepackage{algorithm}
\usepackage{hyperref}
\title{Prefix Sum}
\author{Dimael Rivas Chavez}
\date{July 2024}
\begin{document}
	\maketitle
	\section{Algorithm Explanation}
	Prefix sum can be taken as dp, and one of its applications is in range queries. This algorithm basically saves the sum of range [0,..,$A_i$], in $P_{i+1}$ .
	For example:
	
	Let A be our array and P our prefix sum array.
	
	\begin{algorithm}
		\begin{algorithmic}[0]
			\caption{Prefix Sum build-up}
			\Require $A, P$
			\For {$i = 1$ \textbf{to} $n$}
				\State $P_i \gets P_{i-1} + A_{i-1}$
			\EndFor
		\end{algorithmic}
	\end{algorithm}
	Assuming those arrays are 0-indexed.
	
	\section{Problems}
	\subsection{Optimizations for queries on all sub-arrays}
	Those problems use the same approach:
	
	\href{https://codeforces.com/problemset/problem/1398/C}{Problem 1}
	
	\href{https://codeforces.com/problemset/problem/1996/E}{Problem 2}
	
	The idea here is to make the prefix sum satisfies the property of the query by making that if is $P_{l-1} = P_r$ is true, then $A[l,...,r]$ satisfies the query.
	
	The reason of why it is important to have an equality of $P_{l-1} = P{r}$ (do not know if it can be an extended equation) is that we can store it in a dictionary, so as we passing through r, we can ask in our dictionary all $P_{l-1}$ that are equal to our current $P_r$. Which reduces the complexity of calculating all sub-arrays $O(n^2)$ to $O(n)$ or $O(n\log(n))$ whether you are using a hash map or a tree-structure.
	
	
	
\end{document}